\documentclass{article}
\usepackage{amsmath}
\usepackage{mathrsfs}
\usepackage{amssymb} % for "\mathbb" macro
\usepackage[round]{natbib}
\usepackage{url}
\usepackage{hyperref}
\usepackage[toc,page]{appendix}

\title{Three Factor Seasonal Commodity Price Process}
\author{Jake C. Fowler}
\date{December 2020}

\begin{document}
\newcommand{\+}[1]{\ensuremath{\mathbf{#1}}}

\maketitle

WARNING: THIS DOCUMENT IS CURRENTLY WORK IN PROGRESS

\tableofcontents

\newpage

\section{Introduction}
This paper presents a specific set for parameters for the multi-factor model presented in \cite{Fowler}
such that the model should have similar statistical properties to the three-factor spot price
model presented in \cite{Boogert} the model used in the commercial KyStore gas storage valuation model. 


\section{Forward Price SDE}
The starting point is the SDE (stochastic differential equation) for the forward
price process:

\begin{align}
    \label{eq:forward_sde}
    \frac{dF(t, T)^l}{F(t, T)^l}=\sum_{i=1}^{n^l} \sigma_i^l(T)e^{-\alpha_i^l(T-t)}dz_i^l(t) \\
    \nonumber
    \alpha_i^l \in \mathbb{R}_{\ge 0} \\
    \nonumber
    t \in \mathbb{R}_{\ge 0} \\
    \nonumber
    T \in \{ T_0^l, T_1^l, T_2^l, \hdots | T_j^l \ge t  \} \\
    \nonumber
    \sigma_i^l : \mathbb{R}_{\ge 0} \rightarrow \mathbb{R} \\
    \nonumber
    l \in [1, m]
\end{align}

Where $z_i^l(t)$ follow correlated Wiener processes with correlation $\rho_{i, j}^{x, y}$, i.e.

\begin{equation}
    \mathbb{E}[dz_i^x(t)dz_j^y(t)] = \rho_{i, j}^{x, y}dt
\end{equation}

$F(t, T_j)^l$ is the forward price observed at time $t$, for delivery over the time 
interval $[T_j, T_{j+1})$ of the $l^{\text{th}}$ of $m$ commodity underlyings.


\section{Critique of Three Factor Seasonal Model}
The strength of the three-factor seasonal model is it's parsimony. 
Being able to specify the gas price dynamics using only four parameters is of great help
in allowing users to intuitively see what is driving the extrinsic value of gas storage
being valued. Traders can easily adjust the input
parameters based on their view of the market. For example if a trader take a view that 
the future summer-winter spread volatility is going to be higher than in the historical
period used to calibrate the parameters they can easily bump up the seasonal volatility
parameter when valuing a potential storage deal. For a non-parsimonious model with many parameters
some of which are completely abstract (for example correlation between factors), such
usage would not be practical.

Another example of the is that risk managers can 
create scenario matrices containing storage facility (or portfolio) P\&L based on scenarios 
applied to any of the four parameters. 

% Quote Taleb and compare to Black-Scholes as a parameterisation?

It also allows for a relatively simple and intuitive way of calibrating
model parameters from historic spot and forward prices.
% TODO cite other Kyos paper
The KyStore product is clearly a popular one and The Author believes that this is at least partially due 
to the parsimony of the underlying price process model.

\bigskip
The big downside of this model is that the forward volatility structure implied is completely unrealistic.
It is well known that the 


\bibliographystyle{plainnat}
\bibliography{three_factor_seasonal_model}

\end{document}